

%  while true ; do rsync /home/imaseki/OneDrive/C104/C104Fig/* ./C104Fig/color/ ; python3 ./python/convert_gray_scale.py --input-dir ./C104Fig/color --output-dir ./C104Fig/gray  ; sleep 30 ; done

% 画像の配置は最後に再度確認する

% 推敲TODO
% 節タイトルはパワポ感覚でつけていきたい。

% 語彙統一リスト(基本的にはGoogle Scholarでヒット件数が大きい方と言う決め方)
% Bag of Words
% Self Attention(ハイフン無し)



\documentclass[a5paper,twoside,dvipdfmx]{jsarticle}
\AtBeginDvi{\special{pdf:mapfile haranoaji.map}}
\usepackage[dvipdfmx]{graphicx}
\title{実践PEFT

~ご家庭のGPUでLLM fine-tuning~}
%\author{お椀の底の玉}
\date{}

\usepackage{ascmac}

\usepackage{caption}
\captionsetup[table]{font={scriptsize}}

\usepackage{mathtools}
\mathtoolsset{showonlyrefs=true}

\usepackage[top=30truemm,bottom=20truemm,left=25truemm,right=25truemm,headsep=5truemm]{geometry}

\usepackage{url}

\usepackage{color}

\usepackage{type1cm}

\usepackage{bxpapersize}

\usepackage{booktabs}

\usepackage{listings,jlisting} %日本語のコメントアウトをする場合jlistingが必要
%ここからソースコードの表示に関する設定
\lstset{
  language=Python,
  basicstyle={\ttfamily},
  identifierstyle={\small},
%  commentstyle={\smallitshape},
  keywordstyle={\small\bfseries},
  ndkeywordstyle={\small},
  stringstyle={\small\ttfamily},
  frame={tb},
  breaklines=true,
  columns=[l]{fullflexible},
  numbers=left,
  xrightmargin=0zw,
  xleftmargin=0zw,
  numberstyle={\scriptsize},
  stepnumber=1,
  numbersep=1zw,
  lineskip=-0.5ex,
  showstringspaces=\false
}

\pagestyle{empty}

\usepackage{fancyhdr}  
\pagestyle{fancy}
\fancyhead[RO,LE]{\thepage}
\fancyhead[RE,LO]{\nouppercase{\leftmark}}
\fancyheadoffset[R,L]{1cm}
\cfoot{ }


\usepackage{jumoline}
\setlength{\UnderlineDepth}{3pt}

\begin{document}

% \mcfamily	

\tableofcontents

\newpage

\maketitle

\fontsize{9pt}{16pt}\selectfont

\section{はじめに}



\subsection{本書の目的}




\newpage

\thispagestyle{empty} 

\textcolor{white}{.}
\vspace{\baselineskip}
\vspace{\baselineskip}
\vspace{\baselineskip}
\vspace{\baselineskip}
\vspace{\baselineskip}
\vspace{\baselineskip}
\vspace{\baselineskip}
\vspace{\baselineskip}
\vspace{\baselineskip}
\vspace{\baselineskip}
\vspace{\baselineskip}
\vspace{\baselineskip}
\vspace{\baselineskip}
\vspace{\baselineskip}
\vspace{\baselineskip}
\vspace{\baselineskip}
\vspace{\baselineskip}
\begin{screen}

実践PEFT ~ご家庭のGPUでLLM fine-tuning~

\begin{tabbing}
  0000000 \= 000 \= \kill
  著者 \> お椀の底の玉 \\
  発行日 \> 2024/12/30 (第2版) \\
  発行元 \> ゆるふわ数理研究所 \\
  X ID \> @yurufuwasuuri \\
  印刷所 \> ちょ古っ都製本工房
\end{tabbing}

\end{screen}

\vspace{\baselineskip}
% 本書で使用したスプレッドシートは

% % \url{https://docs.google.com/spreadsheets/d/1-R35DSPx9mZpWCph}

% % \url{WSuliG8pkzbRLOUb9ySijESiDlc/edit?usp=sharing}

% で閲覧が可能です。


\end{document}
\